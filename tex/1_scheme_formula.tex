\section{Схема для $\log(\rho)$ с центральными разностями}
\label{scheme1}

Для автоматического обеспечения условия положительности функции плотности систему дифференциальных уравнений можно преобразовать к виду
\begin{align*}
& \cfrac{\partial g}{\partial t} + \frac{1}{2}\sum_{k=1}^{2}\Big(u_k\cfrac{\partial g}{\partial x_k}+
  \cfrac{\partial u_k g}{\partial x_k} + (2-g)\cfrac{\partial u_k}{\partial x_k}\Big) = f_0,\\
& \cfrac{\partial u_k}{\partial t} + \frac{1}{3}(u_k\cfrac{\partial u_k}{\partial x_k} + 
  \cfrac{\partial u_k^2}{\partial x_k})
  + \frac{1}{2}\sum_{m=1,m\neq k}^{2}\Big( u_m\cfrac{\partial u_k}{\partial x_m}+
  \cfrac{\partial u_m u_k}{\partial x_m} - u_k\cfrac{\partial u_m}{\partial x_m}\Big)
  + p’_{\rho } \cfrac{\partial g}{\partial x_k} = \\
& = \cfrac{\mu}{\rho}\Bigg( \frac{4}{3} \cfrac{\partial^2 u_k}{\partial x_k^2} + 
  \sum_{m=1,m\neq k}^{2}\Big( \cfrac{\partial^2 u_k}{\partial x_m^2} + 
  \frac{1}{3} \cfrac{\partial^2 u_m}{\partial x_k \partial x_m} \Big)\Bigg) + f_k,\ k = 1..s,\\
& p = p(\rho), \quad g = ln\ \rho.
\end{align*}

Сеточную функцию, разностное приближение для плотности $\rho$, обозначим H. Аналогично, разностные аналоги g и u обозначим через G и V. Для поиска численного решения задачи используется следующая разностная схема:
\begin{align*}
& G_t + 0.5 \sum_{k=1}^{2}\Big(V_k \hat G_{\stackrel{0}{x_k}} + (V_k \hat G)_{\stackrel{0}{x_k}} 
  + 2 (\hat V_k)_{\stackrel{0}{x_k}} - G(V_k)_{\stackrel{0}{x_k}} \Big) = f_0,\ x \in \Omega _{\bar h};\\
& G_t + 0.5 \Big((V_k \hat G)_{x_k} + 2 (\hat V_k)_{x_k} - G(V_k)_{x_k}\Big) -\\
& -0.5h_k \Big( (GV_k)_{x_k\bar x_k}^{+1_k} - 0.5(GV_k)_{x_k\bar x_k}^{+2_k}
  + (2-G)((V_k)_{x_k\bar x_k}^{+1_k} - 0.5 (V_k)_{x_k\bar x_k}^{+2_k})\Big) = \\
& \qquad = f_0, \qquad x \in \gamma_k^-, k = 1,\ 2;\\
& G_t + 0.5 \Big((V_k \hat G)_{\bar x_k} + 2 (\hat V_k)_{\bar x_k} - G(V_k)_{\bar x_k}\Big) +\\
& + 0.5h_k \Big( (GV_k)_{x_k\bar x_k}^{-1_k} - 0.5(GV_k)_{x_k\bar x_k}^{-2_k}
  + (2-G)((V_k)_{x_k\bar x_k}^{-1_k} - 0.5 (V_k)_{x_k\bar x_k}^{-2_k})\Big) = \\
& \qquad = f_0, \qquad x \in \gamma_k^+, k = 1,\ 2;\\
& (V_k)_t + \frac{1}{3}\Big( V_k(\hat V_k)_{\stackrel{0}{x_k}} + (V_k \hat V_k)_{\stackrel{0}{x_k}} \Big) +\\
& + \frac{1}{2}\sum_{m=1,m\neq k}^{2}\Big( 
  V_m(\hat V_k)_{\stackrel{0}{x_m}} + (V_m \hat V_k)_{\stackrel{0}{x_m}} - V_k(\hat V_m)_{\stackrel{0}{x_m}} \Big) +\\
& + p’_{\rho }(e^{G})\hat G_{\stackrel{0}{x_m}} = \tilde{\mu} \Bigg( \frac{4}{3}(\hat V_k)_{x_k\bar x_k}
  + \sum_{m=1,m\neq k}^{2}(\hat V_k)_{x_m\bar x_m} \Bigg) -\\
& - (\tilde{\mu} - \mu e^{-G}) \Bigg( \frac{4}{3}(V_k)_{x_k\bar x_k} + \sum_{m=1,m\neq k}^{2}(V_k)_{x_m\bar x_m}
  \Bigg) +\\
& + \cfrac{\mu e^{-G}}{3} \sum_{m=1,m\neq k}^{2}(V_m)_{\stackrel{0}{x_k} \stackrel{0}{x_m}} + f_k,\quad x\in \Omega_{\bar h};\\
& \hat V_k = 0, \qquad x \in \gamma_{\bar h}, \qquad k = 1,\ 2.
\end{align*}

Распишем уравнения схемы в поточечном виде и преобразуем их, приведя подобные слагаемые при неизвестных значениях с верхнего слоя. Получим:
\begin{align*}
& 4 G_{m_1,m_2}^{n+1}  -\frac{\tau}{h_x}  G_{m_1-1,m_2}^{n+1}(V1_{m_1,m_2}^n + V1_{m_1-1,m_2}^n) +\ \frac{\tau}{h_x} G_{m_1+1,m_2}^{n+1}(V1_{m_1,m_2}^n + V1_{m_1+1,m_2}^n)\\
& -\frac{\tau}{h_y} G_{m_1,m_2-1}^{n+1}(V2_{m_1,m_2}^n + V2_{m_1,m_2-1}^n)
 + \frac{\tau}{h_y} G_{m_1,m_2+1}^{n+1}(V2_{m_1,m_2}^n + V2_{m_1,m_2+1}^n)\\
& -\frac{2\tau}{h_x} V1_{m_1-1,m_2}^{n+1} + \frac{2\tau}{h_x} V1_{m_1+1,m_2}^{n+1}
-\frac{2\tau}{h_y} V2_{m_1,m_2-1}^{n+1} +\ \frac{2\tau}{h_y} V2_{m_1,m_2+1}^{n+1} = \\
& =\ 4 G_{m_1,m_2}^n + \tau G_{m_1,m_2}^n 
\Bigg(\cfrac{V1_{m_1+1,m_2}^n - V1_{m_1-1,m_2}^n}{h_x} +  \cfrac{V2_{m_1,m_2+1}^n - V2_{m_1,m_2-1}^n}{h_y}\Bigg) \\
& +\ 4\tau f_0, \qquad x\in \Omega _h
\end{align*}
\begin{align*}
& G_{0,m_2}^{n+1}(2 - \frac{\tau}{h_x}V1_{0,m_2}^n) + G_{1,m_2}^{n+1}\frac{\tau}{h_x}V1_{1,m_2}^n + \\
& + \frac{2\tau}{h_x}V1_{1,m_2}^{n+1} - \frac{2\tau}{h_x}V1_{0,m_2}^{n+1} = 
  2 G_{0,m_2}^n + \frac{\tau}{h_x} G_{0,m_2}^n (V1_{1,m_2}^n - V1_{0,m_2}^n) + 2\tau f_0 +\\
& + \frac{\tau}{h_x} \Big( G_{0,m_2}^n V1_{0,m_2}^n - 2.5 G_{1,m_2}^n V1_{1,m_2}^n 
  + 2 G_{2,m_2}^n V1_{2,m_2}^n - 0.5 G_{3,m_2}^n V1_{3,m_2}^n + \\
& + (2 - G_{0,m_2}^n)
  (V1_{0,m_2}^n - 2.5 V1_{1,m_2}^n + 2 V1_{2,m_2}^n - 0.5 V1_{3,m_2}^n) \Big), \qquad x \in \gamma_k^-
\end{align*}
\begin{align*}
& G_{M,m_2}^{n+1}(2 + \frac{\tau}{h_x}V1_{M,m_2}^n) - G_{M-1,m_2}^{n+1}\frac{\tau}{h_x}V1_{M-1,m_2}^n + \\
& + \frac{2\tau}{h_x}V1_{M,m_2}^{n+1} - \frac{2\tau}{h_x}V1_{M-1,m_2}^{n+1} = 
  2 G_{M,m_2}^n + \frac{\tau}{h_x} G_{M,m_2}^n (V1_{M,m_2}^n - V1_{M-1,m_2}^n) + 2\tau f_0 -\\
& - \frac{\tau}{h_x} \Big( G_{M,m_2}^n V1_{M,m_2}^n - 2.5 G_{M-1,m_2}^n V1_{M-1,m_2}^n 
  + 2 G_{M-2,m_2}^n V1_{M-2,m_2}^n - 0.5 G_{M-3,m_2}^n V1_{M-3,m_2}^n + \\
& + (2 - G_{M,m_2}^n)
  (V1_{M,m_2}^n - 2.5 V1_{M-1,m_2}^n + 2 V1_{M-2,m_2}^n - 0.5 V1_{M-3,m_2}^n) \Big), \qquad x \in \gamma_k^+
\end{align*}
\begin{align*}
& V1_{m_1,m_2}^{n+1}(6 + 4\tau \tilde{\mu}(\cfrac{4}{h_x^2} + \cfrac{3}{h_y^2})) +\\
& V1_{m_1-1,m_2}^{n+1}(-\frac{\tau}{h_x}(V1_{m_1,m_2}^n + V1_{m_1-1,m_2}^n) -\tilde{\mu}\cfrac{8\tau}{h_x^2}) +\\
& V1_{m_1+1,m_2}^{n+1}(\frac{\tau}{h_x}(V1_{m_1,m_2}^n + V1_{m_1+1,m_2}^n) -\tilde{\mu}\cfrac{8\tau}{h_x^2}) +\\
& V1_{m_1,m_2-1}^{n+1}(-\frac{3\tau}{2h_y}(V2_{m_1,m_2}^n + V2_{m_1,m_2-1}^n) -\tilde{\mu}\cfrac{6\tau}{h_y^2}) +\\
& V1_{m_1,m_2+1}^{n+1}(\frac{3\tau}{2h_y}(V2_{m_1,m_2}^n + V2_{m_1,m_2+1}^n) -\tilde{\mu}\cfrac{6\tau}{h_y^2}) -\\
& - 3\frac{\tau}{h_x} p’_{\rho} G_{m_1-1,m_2}^{n+1} + 3\frac{\tau}{h_x} p’_{\rho} G_{m_1+1,m_2}^{n+1} = \\
& =  6 V1_{m_1,m_2}^n + 6\tau f_1 + \frac{3\tau}{2h_y}V1_{m_1,m_2}^n (V2_{m_1,m_2+1}-V2_{m_1,m_2-1}) - \\
& - (\tilde{\mu} - \mu e^{-G}) 6 \tau 
  \big(\cfrac{4}{3h_x^2}(V1_{m_1+1,m_2}^n - 2V1_{m_1,m_2}^n + V1_{m_1-1,m_2}^n) +\\
&  +  \cfrac{1}{h_y^2}(V1_{m_1,m_2+1}^n - 2V1_{m_1,m_2}^n + V1_{m_1,m_2-1}^n)\big) +\\
& + \mu e^{-G} \cfrac{\tau}{2h_x h_y}
  (V2_{m_1+1,m_2+1}^n - V2_{m_1-1,m_2+1}^n - V2_{m_1+1,m_2-1}^n + V2_{m_1-1,m_2-1}^n), \\
& \tilde{\mu} = \mu || e^{-G} ||.
\end{align*}
